\documentclass{article}
\usepackage[left=3cm, right=3cm, top=3cm]{geometry}
\usepackage{amsmath}
\begin{document}
\section*{Theory}
Koopman operator theory is an alternative formulation of dynamical system theory which provides a versatile framework for data-driven methods of high-dimensional nonlinear systems. The theory originated in the 1930s through the work done by Koopman and Von Neuman. Work done in the previous few years has proven the spectral decomposition, introducing the idea of Koopman modes. This theory led to data-driven methods to approximate the Koopman operator spectrum and modes.\\

In a discrete time setting if:
\begin{equation}
x' = T(x)
\end{equation}
is a discrete time dynamical system where $x\in \mathcal{M}$ and $T: \mathcal{M} \to \mathcal{M}$, the the associated Koopman operator $U$ is defined by:
\begin{equation}
Uf(x)=f\circ T(x)
\end{equation}
We call $\phi: \mathcal{M}\to \mathcal{C}$ an eigenfunction of U associated with $\lambda \in \mathcal{C}$ then,
\begin{equation}
U\phi =\lambda \phi
\end{equation}
And in continous time,
\begin{equation}
U^t\phi = e^{\lambda t}\phi
\end{equation}

The eigenfunctions and eigenvalues of the Koopman operator have lots of information about the dynamics. In the previous years a proof of the Koopman Mode Decomposition (KMD) is another outcome of this theory.\\

The Koopman spectrum consists only of eigenvalues, the evolution of observables can be expanded in terms of Koopman eigenfunction denoted as $\phi_j$ where $(j=0,1...)$ and Koopman eigenvalues $\lambda_j$ where $(j=0,1,...)$.\\
The evolution of $f$ is given by:
\begin{equation}
U^nf(x_o) = f \circ T^n(x_o) = \sum_{j=1}^{\infty} v_j\phi_j(x_o)\lambda_{j}^{n} 
\end{equation}
In this decomposition $v_j$ are the Koopman modes associated with the pair $(\lambda_j,\phi_j)$. These modes correspond to components of the physical field characterized by exponential growth and possible oscillations in time.\\

In recent years, various methods have been made to compute spectral properties $(\lambda,\phi,v)$ from data sets. A large fraction of these algorithms are known as Dynamic Mode Decomposition (DMD).

\section{Companion Matrix DMD}
We will call $D$ as the data matrix.\\

1) Define $X=[D_o D_1 \cdots D_{m-1}]$\\

2) Compute $c_j$ values:\\
\begin{align*}
X^\dagger D_m = \begin{bmatrix}
c_o \\
c_1 \\
.\\
.\\
.\\
c_{m-2}
\end{bmatrix}
\end{align*}\\

3) Form the companion matrix:
\begin{align*}
C:= \begin{bmatrix}
0 & 0 &...& 0 &c_o \\
1 &0& ...&  0& c_1 \\
0 &1& ..&. 0& c_2 \\
\vdots& \cdots& \cdots& \cdots & \vdots \\
0 &0& \cdots& 1& c_{m-1}
\end{bmatrix}
\end{align*}\\

4) Get the eigenvalue/vectors from $C$, let $(\lambda_j,w_j)$ be the eigenvalue-vector pair.\\

$\lambda_j$s are the dynamic eigenvalues\

$\phi_j$s are the dynamic modes\\
\begin{align*}
\phi_j=Xw_j
\end{align*}

\section{SVD DMD}
We will call $D$ as the data matrix.\\

1) Form $X=[D_o D_1 \cdots D_{m-1}]$ and $Y=[D_1 D_2 \cdots D_m]$\\

2) Compute the singular value decomposition (SVD) of X:
\begin{align*}
svd(X) = U\Sigma V^*
\end{align*}\\

3) Create $\tilde{A}$ matrix:
\begin{align*}
\tilde{A}=U^*YV\Sigma ^{-1}
\end{align*}\\

4) Get the eigenvalue/vectors from $\tilde{A}W=W\lambda$. Let $(\lambda_j,W_j)$ be the eigenvalue-vector pair.\\

$\lambda_j$s are the dynamic eigenvalues\

$\phi_j$s are the dynamic (projected) modes\\

\begin{align*}
\phi=UW
\end{align*}\\

An alternative eigenvector or Exact DMD Mode $(\phi)$ of $\tilde{A}$ can be given by:
\begin{align*}
\phi = YV\Sigma^{-1}W
\end{align*}


\end{document}